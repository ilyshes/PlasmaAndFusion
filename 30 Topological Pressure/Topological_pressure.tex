\documentclass{article}
\usepackage{amsmath}
\usepackage{amsfonts}
\usepackage{amssymb}

\title{Mathematical Formalism of the Largest Lyapunov Exponent Calculation}
\author{Gemini}
\date{\today}

\begin{document}

\section*{Topological Pressure}

\subsection*{1. Motivation}

Topological entropy \( h_{\text{top}}(f) \) measures the exponential growth rate of distinguishable orbit segments in a dynamical system.  
Topological pressure \( P(\phi) \) generalizes this concept by introducing a continuous \emph{potential function} \( \phi : X \to \mathbb{R} \), 
which allows different orbits to be weighted differently—analogous to weighting microstates by energy in statistical mechanics.

\subsection*{2. Definition}

Let \( f : X \to X \) be a continuous map on a compact metric space \( X \), 
and \( \phi : X \to \mathbb{R} \) be a continuous function.  
Define the $n$-step potential sum along an orbit:
\[
S_n \phi(x) = \sum_{k=0}^{n-1} \phi(f^k(x)).
\]
For given \( n \) and \( \varepsilon > 0 \), let \( E \subset X \) be a maximal \((n,\varepsilon)\)-separated set, 
and define the corresponding partition function:
\[
Z_n(\phi, \varepsilon) = \sum_{x \in E} \exp(S_n \phi(x)).
\]
The \emph{topological pressure} is then
\[
P(\phi) = \lim_{\varepsilon \to 0} 
\left( \limsup_{n \to \infty} 
\frac{1}{n} \ln Z_n(\phi, \varepsilon) \right).
\]

\subsection*{3. Interpretation}

When \( \phi = 0 \), the weights are unity, \( Z_n = N(n,\varepsilon) \), and
\[
P(0) = h_{\text{top}}(f),
\]
so topological pressure generalizes topological entropy.
For nonzero \( \phi \), orbit segments contribute with different weights \( e^{S_n \phi(x)} \),
and \( P(\phi) \) measures the exponential growth rate of this weighted orbit complexity.

\subsection*{4. Variational Principle}

The topological pressure satisfies the \emph{variational principle}:
\[
P(\phi) = \sup_{\mu \in \mathcal{M}_f} 
\left[\, h_\mu(f) + \int \phi \, d\mu \, \right],
\]
where \( \mathcal{M}_f \) is the set of \( f \)-invariant probability measures 
and \( h_\mu(f) \) is the measure-theoretic (Kolmogorov–Sinai) entropy.
The measure \( \mu_\phi \) that attains the supremum is called the \emph{equilibrium state} for \( \phi \).

\subsection*{5. Thermodynamic Analogy}

The correspondence with thermodynamics is summarized below:
\[
\begin{array}{lll}
\text{Energy} & \leftrightarrow & \text{Potential function } \phi, \\[4pt]
\text{Temperature } (T=1/\beta) & \leftrightarrow & \text{Scaling of } \phi, \\[4pt]
\text{Partition function } Z = \sum e^{-\beta E} & \leftrightarrow & Z_n(\phi,\varepsilon) = \sum e^{S_n \phi(x)}, \\[4pt]
\text{Free energy } F = -T \ln Z & \leftrightarrow & P(\phi) = \lim \frac{1}{n}\ln Z_n, \\[4pt]
\text{Entropy } S = -\partial F / \partial T & \leftrightarrow & h_\mu(f) = P(\phi) - \int \phi\, d\mu.
\end{array}
\]

\subsection*{6. Example: Lyapunov Weights}

For one-dimensional maps, choosing
\[
\phi(x) = -t \ln |f'(x)|
\]
leads to the \emph{Bowen equation}
\[
P(\phi_t) = 0,
\]
whose solution in \( t \) gives the Hausdorff dimension of the corresponding invariant set.

\subsection*{7. Summary}

\[
\boxed{
\begin{aligned}
&\text{Topological entropy:} && h_{\text{top}} = P(0), \\[4pt]
&\text{Topological pressure:} && P(\phi) = \text{weighted exponential growth rate of orbit complexity}, \\[4pt]
&\text{Variational principle:} && P(\phi) = \sup_{\mu} \big( h_\mu + \int \phi\, d\mu \big).
\end{aligned}
}
\]


\end{document}
